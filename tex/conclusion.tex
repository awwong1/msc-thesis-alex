% Allow relative paths in included subfiles that are compiled separately
% See https://tex.stackexchange.com/questions/153312/
\providecommand{\main}{..}
\documentclass[\main/thesis.tex]{subfiles}

\onlyinsubfile{\zexternaldocument*{\main/tex/introduction}}

\begin{document}

\chapter{Conclusion}
\chaptermark{conclusion}
\label{chp:conclusion}

In this thesis, I proposed three approaches for the classification of 12-lead \gls{ecg} records.
I demonstrated that \gls{ecg} records can be classified using traditional signal processing and feature extraction techniques in combination with a shallow gradient boosted tree ensemble algorithm (Chapter~\ref{chp:xgbensemble}).
I showcased a deep learning \gls{ecg} record classifier using beat to sequence autoencoders to learn fixed length embeddings from arbitrary length signals (Chapter~\ref{chp:dl_autoenc}).
I proposed a set of experiment configurations, experimenting on the original shallow gradient boosted tree ensemble methodology with labelwise feature pruning and incorporating the autoencoder embedding representations into the classifier inputs~(Chapter~\ref{chp:aencxgb}).

To summarize the three predictions provided in the introduction:
\begin{itemize}
    \item I support my original prediction, showing experimental results indicating that shallow learning boosted decision trees can outperform deep learning autoencoder models on summary classification metrics such as the \emph{PhysioNet/CinC 2020 Challenge} scoring function and overall F-measure.
    \item I support my original prediction, showing that proper regularization of the input feature space and selection of relevant features for the gradient boosted decision tree classifiers are more effective than concatenating autoencoder embeddings for improving the scoring function output.
    \item I refute my original prediction, as naively joining deep learning autoencoder embeddings with manually engineered features for decision tree classifiers does not significantly improve the summary classification metrics in the \gls{ecg} classification task.
\end{itemize}

The gradient boosted decision tree approach, stacked autoencoders approach, and combined approaches described in this thesis are capable of predicting multiple cardiac diagnoses from unstructured, 12-lead \gls{ecg} records.
Specific to the \emph{PhysioNet/CinC 2020 Challenge}, our most prominent approach selects the label-wise top 1000 most important features and autoencoder embeddings from the entire input space of features and trains an XGBoost binary classifier for each of our 27 diagnoses.
This approach achieves an average test split challenge score of $0.4366$, with an overall test split classification accuracy of $0.3068$.

\section{Future Work}
The biggest unrealized gain in the classification of \gls{ecg} records using the \emph{PhysioNet/CinC} provided public data is for a human expert to overread and correct all erroneous labels and discard unusable samples from the available corpus.
This can also be addressed by augmenting the available corpus of data with new \gls{ecg} records that are sourced from known distributions and labelled by trusted cardiologists.
Because the \emph{PhysioNet/CinC 2021 Challenge} extends the current challenge and incorporates a 2-lead classification variant, a replication of this study using a subset of 2 leads is also warranted.

For use as a general \gls{ecg} classifier, the success of the wide and deep transformer architecture proposed by Natarajan \emph{et al}~\cite{natarajan2020CINC-multilabel-ECG} emphasizes the importance of the transformer family of neural networks.
When trained on the raw signal and a selection of manually engineered features, transformers may result in superior classifiers than approaches using convolutions, recurrent neural networks, and shallow classifiers alone.

The classification of \gls{ecg} records in this thesis were limited to the 27 labels defined by the PhysioNet challenge organizers.
Future work should tailor the multi-label multi-class classification task to incorporate a more broad scope of cardiac diagnoses, or explore matching diagnoses to high dimensional embedding space as an alternative to discrete binary classifiers per label.

\end{document}
