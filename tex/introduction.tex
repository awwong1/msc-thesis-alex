% Allow relative paths in included subfiles that are compiled separately
% See https://tex.stackexchange.com/questions/153312/
\providecommand{\main}{..}
\documentclass[\main/thesis.tex]{subfiles}

\begin{document}

\chapter{Introduction}

Heart and cardiovascular diseases are the global leading cause of death~\cite{doi:10.1161/CIR.0000000000000757}.
The most critical tool for the detection and screening of cardiac abormalities is the \gls{ecg}.
The \gls{ecg} is a cardiological test for measuring the electrical activity of the heart.
Using electrodes placed in contact with the skin, the electrical activity corresponding to the heartbeat muscle contractions are recorded.
We need early and automated interpretation of \gls{ecg} signals to screen and 

Here is a test reference~\cite{Knuth68:art_of_programming}.
These additional lines have been added just to demonstrate the spacing
for the rest of the document. Spacing will differ between the typeset main
document, and typeset individual documents, as the commands
to change spacing for the body of the thesis are only in the main document.

\section{Cross-Referencing}\label{sec:crossRef}

Cross-references between child documents are possible using the
\href{https://www.ctan.org/pkg/zref}{\texttt{zref}} package.

\newpage

Text on a new page, to test top margin size.

A sample equation \eqref{eq:test} follows:

\begin{equation}
y = \frac{1}{x^2} + 4 \label{eq:test}
\end{equation}

A sample table, Table \ref{tab:test}:

\begin{table}[h]
    \centering
    \begin{tabular}{r|l}
    \textbf{Right aligned column} & \textbf{Left aligned column} \\ \hline
    This is a right aligned column & Left aligned column
    \end{tabular}
    \caption{A sample table using \texttt{tabular}}
    \label{tab:test}
\end{table}

If there are many acronyms, such as \gls{asa}, and specialized technical terms, consider adding a glossary.
Sample \gls{sampleGlossaryEntry}, and acroynm (\gls{asa}) descriptions are provided above.

\begin{enumerate}[leftmargin=*,nosep]
    \item Sample enumeration
    \item Using the \texttt{enumitem} package.
    \item \eg Foobar
\end{enumerate}

\end{document}