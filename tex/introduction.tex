% Allow relative paths in included subfiles that are compiled separately
% See https://tex.stackexchange.com/questions/153312/
\providecommand{\main}{..}
\documentclass[\main/thesis.tex]{subfiles}

\begin{document}

\chapter{Introduction}

Heart and cardiovascular diseases are the global leading cause of death~\cite{doi:10.1161/CIR.0000000000000757}.
The \gls{ecg}, when correctly interpreted, is the primary tool in our ongoing efforts to detect cardiac abnormalities and screen vulnerable members of our society for heart related issues~\cite{SMULYAN2019153}.
The \gls{ecg} records electrical activity corresponding to the heartbeat muscle contractions using non-invasive electrodes placed in direct contact with the skin.
Although computerized interpretations of \gls{ecg}s are in widespread use, automated approaches have not yet matched the quality of an expert cardiologist reference, leading to poor patient outcomes or even fatality~\cite{BREEN2019}.

\section{Thesis organization}

This work is organized into the following chapters:
Chapter~\ref{chp:background} describes the characteristics of an \gls{ecg}, the dataset of \gls{ecg} records used in our analysis and algorithm training, and the different classification labels that our algorithm predicts probabilities for.
Chapter~\ref{chp:xgbensemble} contains an approach for the classification of \gls{ecg} records using manual feature extraction and a gradient boosted decision tree ensemble.
Chapter~\ref{chp:dl_autoenc} contains a deep learning classification approach using stacked autoencoders to learn an embedding representation of heartbeats and the \gls{ecg} signal.
Chapter~\ref{chp:aencxgb} fuses the autoencoder and decision tree ensemble into one hybrid model and showcases the results in comparison to the prior two methods.
Finally, Chapter~\ref{chp:conclusion} proposes future research directions and concludes the thesis.

% \section{Cross-Referencing}\label{sec:crossRef}

% Cross-references between child documents are possible using the
% \href{https://www.ctan.org/pkg/zref}{\texttt{zref}} package.

% \newpage

% Text on a new page, to test top margin size.

% A sample equation \eqref{eq:test} follows:

% \begin{equation}
% y = \frac{1}{x^2} + 4 \label{eq:test}
% \end{equation}

% A sample table, Table \ref{tab:test}:

% \begin{table}[h]
%     \centering
%     \begin{tabular}{r|l}
%     \textbf{Right aligned column} & \textbf{Left aligned column} \\ \hline
%     This is a right aligned column & Left aligned column
%     \end{tabular}
%     \caption{A sample table using \texttt{tabular}}
%     \label{tab:test}
% \end{table}

% If there are many acronyms, such as \gls{asa}, and specialized technical terms, consider adding a glossary.
% Sample \gls{sampleGlossaryEntry}, and acroynm (\gls{asa}) descriptions are provided above.

% \begin{enumerate}[leftmargin=*,nosep]
%     \item Sample enumeration
%     \item Using the \texttt{enumitem} package.
%     \item \eg Foobar
% \end{enumerate}

\end{document}