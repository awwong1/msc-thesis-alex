% Glossary entry definitions and acronyms
\longnewglossaryentry{sampleGlossaryEntry}{
    name={Glossary Entry},
    text={glossary entry}
}{%
This is a sample glossary entry.

Glossary entry descriptions can span multiple paragraphs.

Remember that glossaries are optional.
The glossary implementation in this template is intended to be simple,
and makes use of only one package, \href{https://www.ctan.org/pkg/glossaries}{\texttt{glossaries}}.
There are more flexible, and fully-featured methods for creating glossaries than the one used here.
}

% \newacronym[description={A sample acroynm description}]{asa}{ASA}{A Sample Acronym}
\newacronym[description={A non-invasive tool for measuring the electrical activity of the heart}]{ecg}{ECG}{electrocardiogram}
\newacronym[description={Diagnosis; a delay or blockage along the right side pathway that electrical impulses travel to trigger a heart beat}]{irbbb}{IRBBB}{incomplete right bundle branch block}
\newacronym[description={Diagnosis; a defect in the anterior half of the left bundle branch, related but distinct from left bundle branch block}]{lanfb}{LAnFB}{left anterior fascicular block}
\newacronym[description={Diagnosis; cardiac pacing stimuli is delivered using external means, such as a pace maker}]{pr}{PR}{pacing rhythm}
\newacronym[description={Diagnosis; the net direction of the depolarization wave of the heart is between $+90\degree$ to $+180\degree$}]{rad}{RAD}{right axis deviation}